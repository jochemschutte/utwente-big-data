\documentclass[a4paper]{report}
\begin{document}
\chapter*{Managing Big Data Proposal}
Thijs van Ede (s1367102) \\
Jochem Schutte (s1004735) \\
Sem Spenkelink (s1375490)
\section*{Introduction (Assignment 3.1)}
The current political landscape is sensitive to public opinion as can be seen by the rise of populism with candidates such as Donald Trump in the USA, Geert Wilders in the Netherlands, and parties such as the FPÖ in Austria. Parties and their candidates adjust their stances and image profile according to election polls. On top of that the voter may change his or her opinion based on preliminary polls as demonstrated by the bandwagon and underdog effect \cite{simon1954bandwagon}. Hence it is important to get a clear overview of the predicted outcome of the election. For this purpose one can take into account traditional sources such as surveys and sampling. However, these methods have as a limitation that they only provide a small subset of eligible voters. To survey a broader audience and get a more complete overview one can take advantage of the readily available data of social media \cite{cukier2013rise}. Multiple sources can be used for the purpose of determining the current political landscape, however this paper limits its data sources to the social media platform Twitter. This research tries to accurately predict the political landscape by performing a sentiment analysis on Dutch tweets in the 6 months before the election in September 2012. We analyze our findings against the actual outcome of these elections and establish the effectiveness of our method. Finally we make a prediction of the election results of the next election in March 2017.
\section*{Literature (Assignment 3.2)}
Similar research to ours has been done on different datasets using different techniques. Due to the large volume of data that is analysed methods have been proposed to use Twitter data streams \cite{wang2012system} as the basis of analysis. This research uses data once and discards it afterwards removing the need for storing vast amounts of data. However, this technique has the disadvantage of being able to process data for only a limited number of times which makes it difficult to analyse underlying patterns after the data has been processed. The above is confirmed by \cite{skoric2012tweets}, where researchers tried to predict election results in the 2011 Singapore general elections. They found that election context affects the accuracy of predictions, as the correlation between tweets and the outcome was weaker in Asia opposed to western Europe. Since there is a lot of media freedom and vocality in the Netherlands, we expect our experiment to be viable. Since it is necessary for the experiment to categorize sentiment correctly, we need to define a keyword list of sentiments and corresponding political parties within these tweets. In other words to define when certain political parties or statements are praised or precipitated. The use of tweets as a basis for sentiment analysis has been shown by Pak et al. \cite{pak2010twitter} who built a sentiment classifier based on multinomial Naïve Bayes using N-grams and POS-tags as features. Furthermore, analysis of tweets during the 2014 FIFA World Cup \cite{yu2015world} proved that people use social media as twitter to convey their emotions. This concept can also be applied to politics, Bermingham et al. \cite{bermingham2011using} show that sentiment analysis is a useful tool for determining the popularity of different political parties. However determining how to implement this sentiment analysis may take some effort because there is no general solution to qualify the sentiment of a text. Instead many methods exist that may evaluate texts on different levels and each application requires a different analysis method \cite{liu2012sentiment}. Additionally, Bermingham et al. argue that the volume of data about political parties and their candidates also is a good indicator of election results since popular candidates will get mentioned more often and because sentiment analysis can be difficult due to the difference between actual sentiments of people, and the sentiments they voice to the public.
\section*{Methodology (Assignment 3.3)}
In order to predict the political landscape we use all Dutch tweets from six months prior to the election. Due to the vast amount of analyzed data, this data is processed using a Spark cluster providing enough resources to perform the computation. The historic data contains loads of tweets that do not relate to the elections, hence these need to be filtered out. The data will be filtered based on whether the tweet’s text contains data from a politics dictionary. This dictionary contains the names of all political parties active during that period in the Netherlands, as well as the names of all candidates from said parties. Additionally the dictionary includes political statements, which are found by analyzing all the hashtag keywords tweeted by the aforementioned political parties. Once the relevant tweets have been extracted they are grouped according to the corresponding party. In the next phase, we apply three techniques for analyzing the tweets. The first technique is aimed at determining popularity based on the number of times a tweet mentions a political party, its candidates or its stances. This gives a baseline in which no sentiment analysis is involved. The second technique tries to determine the effect of sentiment in tweets for all political parties, and are thus scored based on a sentiment analysis. This analysis is based on a sentiment dictionary (constructed from words found in https://gist.github.com/FrankHouweling/7fce4b89da4357744054 ,\\ https://nexttalent.nl/etc/media/files/motivatiebrief-hulpwoordenlijst.pdf i.a.) \\
where positive words increase the sentiment score, negative words decrease the score, and negation of words is taken into account, e.g. ‘not good’ will receive the negative score of ‘good’. The scores of all tweets are accumulated per party resulting in an overall score of that party. The third technique argues that people tweeting negative sentiments about a party are unlikely to vote for that party. Hence we remove all negative sentiments, i.e. the tweets with a negative sentiment score, from the second analysis technique and only accumulate the remaining - positive - scores of tweets per party. Using these three techniques we obtain three accumulations per party. For each of these accumulations we map the scores to the representation of the Dutch parliament, i.e. we divide the scores over the 150 available seats in the Dutch House of Representatives. The number of seats assigned to each party according to our analysis is compared to the actual outcome of the 2012 Dutch elections to determine the effectiveness of the three applied techniques.
\section*{Hypothesis (Assignment 3.4)}
The intended end result of our study is to produce a map depicting the distribution of parties in the Dutch House of Representatives as result of the 2012 national elections. This map should be accurate and solely based on the Twitter communications leading up to the election. Important for the legitimacy of the developed tool is its reusability. The tool should be able to be used for other Dutch elections (or similar international elections) with minimal adjustment of the tool. Preferably it should be possible to predict other elections by only inputting a different dataset and sentiment dictionary.
Bibliography
\bibliography{bibliography}{}
\bibliographystyle{plain}
\end{document}